\documentclass[11pt]{article}

\usepackage[margin=1in]{geometry}
\usepackage{amsmath, amssymb, amsfonts}
\usepackage{physics}
\usepackage{graphicx}
\usepackage{bm}
\usepackage{hyperref}

\title{Theory of CHSH Violation for a Rotated Bell State}
\begin{document}
\maketitle{}
\begin{abstract}
This document presents the theoretical background for the numerical
simulation that evaluates the CHSH quantity for a maximally entangled
Bell state under local rotation operations. The code computes the
expectation value of the CHSH correlator as a function of rotation
angles and illustrates violation of the classical CHSH bound.
\end{abstract}

\section{The Bell State}
The simulation uses the well-known Bell state
\begin{equation}
    \ket{\psi_{+}}
    = \frac{1}{\sqrt{2}}\left( \ket{00} + \ket{11} \right),
\end{equation}
which is a maximally entangled two-qubit state.

In the code this is produced via
\[
\ket{0} = \texttt{basis(2,0)}, \qquad
\ket{1} = \texttt{basis(2,1)},
\]
and
\[
\ket{\psi_{+}} = \frac{1}{\sqrt{2}}
\left( \ket{0}\otimes\ket{0} + \ket{1}\otimes\ket{1} \right).
\]

\section{Local Rotation Operator}
Each qubit is independently rotated by a single-qubit unitary of the form
\begin{equation}
    R(\theta)
    = \cos\theta\,\mathbb{I}
      + \sin\theta\left(\ket{1}\bra{0} - \ket{0}\bra{1}\right).
\end{equation}

This operator is equivalent (up to phase) to a rotation around the $y$-axis
of the Bloch sphere:
\begin{equation}
    R(\theta)
    = \exp\left(-i\theta\,\sigma_y\right).
\end{equation}

In the code this is implemented as
\[
R(\theta) = \cos\theta\,I + \sin\theta\,( \ket{1}\bra{0} - \ket{0}\bra{1} ).
\]

\section{Measurement Observable}
Both qubits are measured in the $\sigma_z$ basis.  
The joint observable is
\begin{equation}
    A\otimes B = \sigma_z \otimes \sigma_z.
\end{equation}

Expectation values are computed after rotating the Bell state:
\[
E(\theta_i, \theta'_j)
    = \matrixel{\psi_{+}}{
        R(\theta_i)^{\otimes 2\dagger} \,
        (\sigma_z\!\otimes\!\sigma_z)\,
        R(\theta'_j)^{\otimes 2}}
       {\psi_{+}}.
\]

\section{CHSH Quantity}
The CHSH expression is defined as
\begin{equation}
S =
E(a,b)
+ E(a,b')
+ E(a',b)
- E(a',b').
\end{equation}

Classically, any local hidden-variable theory obeys the bound
\[
|S| \le 2.
\]

\subsection*{Angle Choices}
Your simulation uses
\[
a = \frac{t}{2}, \qquad
a' = -\frac{t}{2}, \qquad
b = 0, \qquad
b' = t.
\]

For each value of the parameter $t$, the code computes $S(t)$.

\section{Violation of CHSH Inequality}
The plot produced by your script shows:

\begin{itemize}
    \item The CHSH value $S(t)$ as a function of the angle parameter.
    \item Shaded regions between $|S|=2$ and $|S|=5$ emphasizing the
          quantum-violating region.
\end{itemize}

Whenever
\[
|S(t)| > 2,
\]
the Bell inequality is violated, demonstrating that the correlations
generated by the Bell state cannot be explained by any local realistic
theory.

\section{Summary}
This simulation evaluates the CHSH correlator for a maximally entangled
Bell state subjected to local rotations. The numerical results reproduce
the theoretical prediction: the CHSH quantity oscillates and periodically
exceeds the classical bound of $2$, confirming quantum nonlocality.
The method implemented corresponds directly to the theoretical formulation
of Bell tests using local unitary rotations and $\sigma_z$ measurements.

\end{document}
